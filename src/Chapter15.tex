\begin{problem}{15.1}
  Jill manages a long-only technology sector fund. Joe manages a risk-controlled, broadly diversified core equity fund. Both have information ratios of 0.5. Which would experience a larger boost in information ratio by implementing his or her strategy as a long/short portfolio? Under what circumstances would Jill come out ahead? What about Joe?
\end{problem}

\begin{proof}[Solution]
\end{proof}

\begin{problem}{15.2}
  You have a strategy with an information ratio of 0.5, following 250 stocks. You invest long-only, with active risk of 4 percent. Approximately what alpha should you expect? Convert this to the shrinkage in skill (measured by the information coefficient) induced by the long-only constraint).
\end{problem}

\begin{proof}[Solution]
\end{proof}

\begin{problem}{15.3}
  How could you mitigate the negative size bias induced by the long-only constraint?
\end{problem}

\begin{proof}[Solution]
\end{proof}

